% Created 2017-02-08 Wed 23:26
% Intended LaTeX compiler: pdflatex
\documentclass[11pt]{article}
\usepackage[utf8]{inputenc}
\usepackage[T1]{fontenc}
\usepackage{graphicx}
\usepackage{grffile}
\usepackage{longtable}
\usepackage{wrapfig}
\usepackage{rotating}
\usepackage[normalem]{ulem}
\usepackage{amsmath}
\usepackage{textcomp}
\usepackage{amssymb}
\usepackage{capt-of}
\usepackage{hyperref}
\author{Noel Song}
\date{\today}
\title{}
\hypersetup{
 pdfauthor={Noel Song},
 pdftitle={},
 pdfkeywords={},
 pdfsubject={},
 pdfcreator={Emacs 25.1.1 (Org mode 9.0.2)}, 
 pdflang={English}}
\begin{document}

\tableofcontents

\section{Week 1}
\label{sec:orgf630056}
\subsection{Intro Problem}
\label{sec:orga78d26b}
Given a text file as input. Generate a list of words sorted according to its
frequency in the input file.

\begin{itemize}
\item Literate Programming

\item Professor simple method
\begin{verbatim}
tr -cs a-zA-Z [\n]  |  sort  | uniq  -c  | sort -nr
\end{verbatim}

\item Parallelism on the Above Method
In a four core matchine it is possible for true parallelism. However, sort
can only output after finishing processing all its input.
\end{itemize}
\subsection{Language popularity}
\label{sec:orgbc87239}
\begin{enumerate}
\item Java
\item C
\item C++
\item C\#
\item Python
\item VBNET
\item js
\item Perl
\item assm
\item PHP
\end{enumerate}

\subsection{Course description}
\label{sec:org2cae0e3}
\begin{itemize}
\item Core of this course
\begin{itemize}
\item Principles and \textbf{limitations} of programming models.

\item Notations for these models
How to design, use and support for.

\item Methods to evaluate the strength and weakness of these notations in
various contexts.
\end{itemize}

\item Current reading assignment: Ch 1-3, 5,7,9

\item Grading:
\begin{itemize}
\item 40\% final exam; open book + notes; closed computer

\item 20\% midterm:

\item 40\% HW: 6 - each 5\%; 1 - 10\%

\item Late penalty: same as 35L; due on 23:55 on the due date
\end{itemize}

\item Topics
\begin{itemize}
\item Theory behind programming languages

\item Language design

\item Syntax

\item Semantics

\item Functions

\item Names

\item Types: descriptions of values

\item Control: Art of letting your program use CPU

\item Objects

\item Exceptions: Sort of control

\item Concurrency

\item Scripting: like shell, python

\item Exercises on Ocaml, Prolong, Java, Python
\end{itemize}

\item Some questions
\begin{enumerate}
\item Why are there so many programming languages?
Why cna't we just have one? There is a paper about "NExt 700 programming
language".
Why not only a broad-spectrum approach?

\item Why software is so slow? (in lisp)
Try C++ instead for fast memory access but loss reliability somehow.
\end{enumerate}
\end{itemize}

\subsection{Reasons to prefer one language to another}
\label{sec:org5d833c0}
\begin{itemize}
\item Popularity
\item Support from compilers, debuggers, etc.
\item Similarity to existing languages
\item Simplicity
\item Versatility: How widly can we apply this language
\item Performance
\begin{itemize}
\item CPU
\item Memory
\item IO (disk/flash)
\item Network
\end{itemize}
\item Scalability: how well the language is for larger and larger program.
\item Reliability
\item Convenience: How easy to use? (i++ in C)
\end{itemize}

We have competing goals, like similarity is in contradiction with convenience.

\begin{itemize}
\item Orthoganality
From mathematic concept, there are independent axis which means change in one
axis does not impose changes on another axis. Once we have multiple features,
people do not need to worry about some other features.

Example: In C, a function can return any kind of types
\begin{verbatim}
int f();
char g();
struct s h();
\end{verbatim}
However, we cannot return an array (or a function). One argument is that copy
is potentially expensive.
Then what about a structure? We can have a structure that is really big and
that works. 
They want to emphasize that arr and pointer are the same thing.

Consider you want to write a code return some type t defined
elsewhere. However, that is not guaranteed feasible (what if that t is an
array type or function type?). Therefore, C kind of lack of orthoganality.

\item Safety(Important subset of reliability)
Does not cause really bad consequence.

\item Concurrency for multi-threaded program
\item Exceptions handling
\item Mutability (Successful language evolve)$\backslash$
\end{itemize}
We want language that can change over time.

Example: A language on 1971 run on PDP-11
two int add
16 KB RAM
1.2 us memory

Anyway, on this machine, memory access if considerably faster than
cpu. Therefore, a language tuned for this machine has easy access to memory. 

Nowadays, cpu speed is 100x faster than memory access.

Example: Varying number of args for function.
  In C there is some called Obj.
  Obj arr[7];
  arr[4] = \ldots{}
  arr[1] = \ldots{}
  Foo(7, arr); 
  What we really want: foo(x, yz, w, u, v) but we have to fill those variables
  in array. 

We cannot realize this exactly but we can approach this as much as we can.
\begin{verbatim}
#define ELTS(a) sizeof(a)/sizeof(*(a))
#define CALLMANY(f, args) (f)(ELTS(args), args)

#define CALLN(f, ...)  CALLMANY(f, (Obj[]{__VA_ARGS__}))

Obj args[7];
// Some Assignemtns
//CALLMANY(Foo, args);
CALLN(Foo, x, y,z ,u, v, a, b ,c);
\end{verbatim}

\begin{verbatim}
#define IF if(
#define THEN ){
#define ELSE } else {
#define FI }
\end{verbatim}

These are examples of mutability. Can the program evolve with the changing
hardware (outside world). Can it change according to the idea of the
programmer.

\subsection{Syntax}
\label{sec:orga024d6d}
\begin{itemize}
\item whether a program is valid????
\begin{verbatim}
int x = "abc"; // No syntax rule indicates that this is invalid
int main(void) {return ""[1]; } // Stil, no syntax rule indicates that this is not valid.
\end{verbatim}
Above programs are definitely wrong but not because of wrong syntax.

\item whether a program will compile????
Consider the first example. It will not compile not because its syntax.

\item \textbf{A syntax is the form independent of meanings}
In syntax we just care about its form. We do not care about what does it mean.

\item Syntax
"Clolorless greed ideas sleep furiously." ===> Good syntax, nonsensee
semantics.

\item Natural Language Syntax
"Ireland has leoperd galore. "--P.Egger  ===> Bad syntax but meaning is
clear

In real life the meaning is important. However, in the world of
programming, syntax is critical.

"Time flies." 'Time' chould be none and verb, so does 'files'. This is the
problem of ambiguity.

\item Programming language syntax: Reasons to prefer one over another
\begin{itemize}
\item Inertia: Use syntax people have already get used to.
a + b OR a b +

\item simple and regular
Pick syntax that lets us say things easily.
Some claim postfix notation is simpler and regular.
In C, (a+b)*c. With postfix, a b + c * ==> No parens.

The first reason prefers normal syntax

\item It's unambiguous
\begin{verbatim}
a---b;
// a-----b; does not work
a--- --b;
\end{verbatim}
Why a-----b does not work?
Because the token analyzer (tokenizer) reads from left to right and it reads
greedily. The above line of code is intepreted as (a)(--)(--)(-)(b).
\textbf{Greedy means it is going to keep gluing characters together}
\end{itemize}
\end{itemize}
\begin{itemize}
\item Tokenization
comment: ??

\item Language without keyword: no reserved word
Why we have language like this? ==> mutability? backward compatibility?
Note that for C/C++ it is painful to improve the language.
\end{itemize}

Clear syntax is one where visualization skill "work"
Example
do \{
   x/=10;
\} while(x > 10);

do \{
  x/=10
\} until(x <= 10)

if( i > 0 \&\& i < 1000) // Not what we want
if(0 < i \&\& i < 1000)

\section{Week 2}
\label{sec:orgb0b9746}
\subsection{On Context free language}
\label{sec:org2ee43c3}
\subsubsection{Some intro}
\label{sec:org6812f68}
\begin{itemize}
\item Sentence= finite sequence of tokens(terminal symbols)

\item Token is from a set of finite elements.
\begin{itemize}
\item Identifier: A special kind of token.
\item Some tokens have special information associated which is important for the
language's semantics.
\item A sentence must be finite
\item Set of tokens must be finite: Important for analyzing of the language.

Example: L = \{a\(^{\text{nb}}\)\(^{\text{n}}\) | n >= 0\}
	 This language has infinite token set.
	 This language does not scale well.
\end{itemize}

\item Nonterminal
Short for a (finite) sequence of tokens, than can be part of a sentence.

\item Grammar
\begin{itemize}
\item set of tokens(terminal symbols) and a finite set of nonterminal symbols.

\item Start symbol: a nonterminal symbols

\item finite set of rules

Example: <nonterminal> -> finite sequence of symbols (either a terminal or
nonterminal)

\item The compiler is going to figure out whether a sentence is a valid program language.

\item Recursive grammar: A symbol is both on the RHS and LHS

Example: Internet RFS 5322 (request for comment)
comtains a grammar for email headers.
\begin{itemize}
\item Subject: <any sequence of chars, except newline>
Grammar of subject content:
      subject contents -> empty
      subject contents -> subject\(_{\text{contents}}\) CHAR
      (subject content -> CHAR subject\(_{\text{contents}}\))

Different grammar can produce the same sentence.

\item MessageID: <xxx.xxx@xx.xx.xx>
Grammar of MessageID:
msg-id = "<" word *<"."word> "@" atom * ("." atom) ">"

\begin{itemize}
\item Some meta-notation like "*" (not strict BNF but EBNF)
EBNF just give us some convenience
BNF <==> EBNF
dotword -> <empty> | . word dotword vs. *<"."word>
BNF is better for parser
\end{itemize}

\item Some other grammar
atom = 1 * <any CHAR except special , SPACE and CR>  
specials = ()<>@  
word = atom | quotestring
CTLS = $\backslash$000 $\backslash$037 $\backslash$177
quotestring = <"> *(qtext | quoted-pair) <">
qtext = <any CHAR except " $\backslash$ newline>
quoted-pair = "$\backslash$" CHAR

\item Possible trouble
\begin{itemize}
\item Infinite long string
Fixed by standard other than grammar

\item Control symbol in message
\end{itemize}

\item Use RegExpr: Turn a specification of a language to a program
qp = '$\backslash$\.'
qt = '[\^{}$\backslash$"]'
ps =

Regular expression can be used to write grammar with meta-symbols and terminal symbols.
However, regExpr cannot count.
Note that recursion is not the limitation.
Example: T -> abT is possible (tail-recursion)
	 T -> aTb is not possible to convert (not tail-recursion)
	 (tail-recursion rule)
\end{itemize}
\end{itemize}

\item ISO Standard for EBNF: Decreasing in precedence
\begin{itemize}
\item "token" or 'token'
\item\relax [ option ]
\item \{ repetition \}
\item (* comment *)
\item X* repetition of X
\item X-Y: X except for Y
\item X,Y: concatenation
\item X | Y: OR
\item X = expr;
\end{itemize}

\item Define EBNF formmally
syntax = syntax rule, \{syntax rule\};
syntax rule = meta id, '=', defines list, ';' ;
definitions list = defn, \{ '|', defns\} ;
Note: Define itself through itself

\item Problems of Grammar
\begin{itemize}
\item Large grammar is hard to understand
\begin{itemize}
\item Syntax chart
\end{itemize}
\item Nonterminal used but not defined (bug in grammar)
Or better described as "non-productive"
\item Nonterminal are defined but not used
Or better described as "non-reachable"
Cannot reach it from ths start symbol
This applies recursively, consider the following with start symbol A
W -> W A
A -> \emph{some other rules} | <empty>

\item Ambiguity
Has multiple way to parse the same input
By making the program easy to parse, you may introduce ambiguity

\begin{itemize}
\item Associativity
E -> E+E
E -> E-E
E -> ID
E -> (E)

Has two ways to parse the sentence "a - b - c"

\begin{itemize}
\item Problem: Is it possible to write a program to test the ambiguity of a
grammar just by looking at the grammar itself?

It is too hard. However, we still want to debug the grammar. 比如debug普通
的程序很难,但是我们还是会去debug并且有所收获。Solve the general problem
of ambiguity is hard, but we are looking for a practical partial solution.

\item General Idea: prohibit the part we do not want. Ambiguous grammar is too
\end{itemize}
general. Consider the above grammar, is too general. Consider the second
rule, the second E cannot be a plus/minus operation.

\begin{itemize}
\item Change: result in a more complicated parse tree
\end{itemize}
E -> T
E -> E+E (This is stil a problem because E is allowed to recurse on both side)
E -> E-T
T -> ID
T -> ( E )

\begin{itemize}
\item Why we must have (a+b)+c?
\item Arithmetic overflow: Some order of the operation results in overflow
\item Floating point rounding problem
\item Side effects?
\item Unsigned arithmetic: Wrap around give different answer
(a+b)+c if a and b are unsigned int but c is unsigned long
\end{itemize}

\item Precedence

E -> T
E -> E + T
E -> E*T   ==> E -> F * T
	   ==> T -> F
T -> ID    ==> F -> ID
T -> ( E ) ==> F -> ( E )

\begin{itemize}
\item Infix grammar has trouble of precedence
\end{itemize}

\item Another ambiguity in C
\begin{itemize}
\item stmt: 
;
break;
continue;
expr;
return expr; (no parens)
while (expr) stmt
do stmt while (expr); (we have no parens)

\begin{itemize}
\item Some modification:
do stmt while expr; ==> This is not going to cause trouble)
while expr stmt     ==> This may cause ambiguous because parser
may combine expr and stmt together:
while i<1 *c / while i<1 -c;

The parens for expr in do-while is not going to solve ambiguity
but it is still there because we want consistency.
\end{itemize}

\item Some other statement
\{ stmt or declaration list \}
for(expr; expr; expr;) stmt
if (expr) stmt
if (expr) stmt else stmt
switch (expr) stmt

\begin{itemize}
\item Dangling "else" problem
fix: stmt: if (expr) stmt | stmt1 
stmt1 has everything except if (expr) stmt
\end{itemize}
\end{itemize}
\end{itemize}

\item Too much details 
\begin{itemize}
\item because of ambiguity
\begin{itemize}
\item Abstract syntax and concrete syntax
Let the parser deal with known ambiguity issue directly instead of writing
complex grammar rule.
\end{itemize}
\item because of attempts to cover too much
IntE -> IntE + IntE | IntE * IntE | IntID| Inte
StrE -> StrID | StrLiteral | StrE \^{} StrE

Note: The above is rarely done in real world language because there are
infinite number of types. Generally grammar is only good at deal with
syntax but not semantics.
\end{itemize}
\end{itemize}

\item Expressions problems

\begin{itemize}
\item user-defined operators (not the c++ way because we must pick the existing
operator and give it new meaning)

\begin{itemize}
\item Prolong: defining new operator
\end{itemize}
:-op (700, xfx, [=, <=, >=,])
:-op (500, yfx, [+,-])
:-op (400, yfx, [*, /])
:-op (200, xfy, [**])
:-op (1200, xfy, [<--])

\begin{itemize}
\item 500 is the precedence
\item y means same precedence is ok
\item x means same precedence is not ok

\item Note that ambiguity is not possible because "yfy" is not allowed.
\end{itemize}

\item Side effect(especially in C/C++): a = f(x) + g(x);
a = a++ -a++;
We cannot determine the order of executing because compiler will optimize
the code.

Example of Professor:
\begin{verbatim}
double stack[1000];
double *top = &sstack[1000];
#define PUSH(x) (*--top = (x))
#define POP()   (*top++)
switch(op) {
 case MULTI:
   PUSH( POP() * POP());
   break;
 }
\end{verbatim}
*--top = (*top++) * (*top++);
\end{itemize}
\end{itemize}


\section{Week 3}
\label{sec:org42e3059}
\subsection{Functional programming}
\label{sec:org65cdde2}
\begin{itemize}
\item Functional programming motivation
\begin{itemize}
\item Clarity
\item performance
\end{itemize}
\end{itemize}

\subsubsection{Clarity:}
\label{sec:orgd5c2ac2}
\begin{itemize}
\item Take advantage of mathematic notation
We want to build on mathematic traditions.
\begin{itemize}
\item a + b = b + a
May not true in C due to side effects
(a+=10) + (a-=10)
\item E - E = 0
what if E is getchar() which return different char each time.
\item i = i + 1 is definitely false mathematically
\end{itemize}
\end{itemize}

\subsubsection{Performance}
\label{sec:org18fa146}
C or Java are designed for Von Neuman architecture(loading and store)

\subsubsection{One success story (empty)}
\label{sec:org6484c3a}

\subsubsection{Function: What is a function}
\label{sec:org5aaf39a}
\begin{itemize}
\item CS: a chunk of code with a name
This is how we think when we are implementing the function.

\item Math: A mapping from a domain to a range
This is how we think when we are using a function.

\item Side Effect: Evaluation is no longer imposed by sequencing but only by function calls.
\begin{verbatim}
a = f(x);
b = g(y);
c = h(z);
e(a, b, c);
\end{verbatim}
The above code might have hidden dependency so the sequential relationship
must be as written.

\begin{verbatim}
a = b + f(x) - b;
\end{verbatim}
In C/C++ or Java, b cannot be eliminated.

\item Referential transparency: Variables with the same name always have the same
value.
\begin{verbatim}
let a = b + (f x) - b
\end{verbatim}
\end{itemize}

\subsubsection{Higher Order Function (Functional forms in math)}
\label{sec:org38192ea}
\begin{itemize}
\item Functions that take function as argument
"Build function on some other functions".
\begin{itemize}
\item Summation operator
\item Integration operator
\item Function composition operator "o"
\end{itemize}
\end{itemize}

\subsubsection{Ocaml}
\label{sec:org1519363}
\begin{itemize}
\item Compile time type checking (like C/C++, Java)
\begin{itemize}
\item Type inference
Do not need to write all types down. The types are inferred from the
context.
\end{itemize}

\item Garbage Collection
No need to worry about storage management.

\item Good support for higher order function

\item Name and Types are really important in Ocaml

\item Pattern matching
\begin{verbatim}
let cons (x,y) = x::y;;
(* The above is equivalent to the following*)
     (fun a -> 
       match a with
       | (x,y) -> x::y)
\end{verbatim}
\end{itemize}

\subsubsection{Functions}
\label{sec:orgd7761ad}
\begin{itemize}
\item Simple functions
\begin{verbatim}
(fun a -> a);;
-: 'a -> 'a = <fun>

(fun () -> ());;
-: unit -> unit
\end{verbatim}

\item Higher order function
\begin{verbatim}
(fun x -> match x with
	  | 0 -> (fun x -> -x)
	  | _ -> (fun x -> x * 2));;
-: int -> (int -> int) = <fun>
\end{verbatim}

\begin{itemize}
\item Note that '->' is right associative
\item Function is left associative
\end{itemize}

\item Recursion
\begin{itemize}
\item Fib
\begin{verbatim}
let rec fib n =
match n with
| 0 -> 1
| 1 -> 1
| n -> fib(n-1) + fib(n-2)
\end{verbatim}
\end{itemize}

\item Currying function: \textbf{All the functions are curried}
\begin{itemize}
\item Max element in list
\begin{verbatim}
let rec maxlist =
function 
| (x::l) -> let m = maxlist l in if x < m then m else x
| [] -> 0;;
\end{verbatim}

\item Reverse
\begin{verbatim}
let rec reverse= function
  | [] -> []
  | (x::l) -> (rev l) @ [x]
\end{verbatim}
\end{itemize}
\end{itemize}


\subsection{Reading}
\label{sec:org0067397}
1-9, 13, 15, 17
\subsection{HW}
\label{sec:org40272d1}
\subsubsection{Theory}
\label{sec:orgce00f6e}
\begin{enumerate}
\item Finite State Automata
\item Push Down Automata = FSA + Stack
\item Turing Machine: FSA dealing with infinite term.
\end{enumerate}

\subsubsection{Parsing a CFG}
\label{sec:org985bd24}
\begin{enumerate}
\item Recursion
Grammar rules are recursive.
\item Concatenation
Suppose S -> T U and we know how to build a matcher for T and U
matcher = acceptor -> token list -> bool

\item OR (disjunction)
\begin{itemize}
\item lookahead
S -> a T b
S -> c U d
Check if it is a or c

However,
S -> a T b
S -> a U d

But we can change it into the following form to use lookahead
S -> a S'
S' -> T b
S' -> U d

But if the "T" is "if then"  ???

\item Create a choice point while parsing
First parse a T b, if fail then parse a U b
BUt need memory to record information

\item Stack
Note that function call is build-in stack.

\item acceptor(token list -> bool)
Takes a string of tokens and indicates correctness
\begin{verbatim}
function | ")"::_ -> true | _ -> false
\end{verbatim}
\end{itemize}
\end{enumerate}

\subsection{Translation into machine code}
\label{sec:org6a4769d}
\begin{itemize}
\item Consider the following code
\begin{verbatim}
int main()
{
  return ! getchar();
}
\end{verbatim}
This piece of code will not compile because getchar() has not been
declared. We need to include stdio.h

\begin{verbatim}
#include <stdio.h>
int mainn() { return !getchar();}
\end{verbatim}

\item Translation
\begin{itemize}
\item Preprocessor: a kind of text expansion.
\begin{verbatim}
extern int getchar(void);

int main() {return !getchar();}
\end{verbatim}

\item Convert program text into sequence of tokens
Each token is represented by a byte.

\textbf{lexing}: extra information associated with the token. For example, the
difference between "main" and any other identifier.

\item Parsing: Parse tree
\item Semantics analysis
\begin{itemize}
\item Type checking
\item Scope checking
\end{itemize}
\item Intermediate code generation
main:
     push \&getchar
     push 0
     call 
     not
     ret
\item Code generation x86\(_{\text{64}}\)
main:
     call getchar
     tstl \%eax, \%eax
     setn \%eax, \%eax
     ret
\end{itemize}

\item Unix design philosophy (software tools)
Break a procoess into multiple independent pieces.
Ensures independence but increase overhead.

\item Smalltallc Design Philosophy (IDE)
\end{itemize}


\section{Week 4}
\label{sec:org87a8f35}

\subsection{Compiler vs. Intepreter}
\label{sec:orgc70d434}


\subsection{Hybrid Approach: JVM}
\label{sec:org0ccf83b}

Foo.java -> java compiler -> Foo.class(bytecodes)
\begin{center}
\begin{tabular}{l}
\\
\end{tabular}
\end{center}
$\backslash$/
Java(intepret + profiler(for performance))
hotpot -> (another compiler) -> machine code
(this method is expensive so we only do it for "hotspot")

\subsubsection{Profiling}
\label{sec:orgc5ee857}
\begin{itemize}
\item statistical: sample hw ip 100 times/s => make a histogram
We can know the "hotspot" of our program

\item Insert counting code (gcc -pg -fprofile -\ldots{})
finer graine performance measrue

\item Once we know the "hotspot" we translate it into machine code

\item Cause nondeterministic performance

\item Hard to debug on the lower level compiler
\begin{itemize}
\item Nondeterministic
\item Generally we will never see the machine code??
\end{itemize}
\end{itemize}


\subsection{Linking}
\label{sec:org30332c0}

\begin{itemize}
\item Loading time dynamic linking (load code before main start)
Library will be dynamically linked as part of initialization process.

\item Running dynamic linking (use special function to call)
Self modifying code
\begin{itemize}
\item Pros: Flexibility
\item Cons: Library version conflict: some module need version 1 but some other need version 2
\item Cons: Worry about unstructed code
Once we allow dynamic linking, our program may execute some suspicious code.
\end{itemize}
\end{itemize}

\subsection{Type}
\label{sec:org4aa901e}
\begin{itemize}
\item Why we need type
\begin{itemize}
\item Help us understand the code
when we look at an id x, we know which operations are valid and which are
not.
\item Reliability/Redundancy
\begin{verbatim}
string x;
x = x + 1; // ???
\end{verbatim}
Catch trivial mistakes early
\item Performance
Type helps compiler generate better code
\end{itemize}

\item Strongly Typed: All operations are type checked, you cannot escape
C/C++ is not strongly typed.

\item Primitive vs. Constructed types
\begin{itemize}
\item Primitive types are built in while structure type are built by user.
\item array of C/C++ are constructed types
\end{itemize}

\item What is type? Is it some data structure? How exposed is the implementation of
a type?
\begin{itemize}
\item Exposed Type
User know details of the type: integral type

\item Abstract Type
User only konw what operation can be done to that type
check NaN
\begin{verbatim}
bool isnanf(float x)
{ return x!=x; }
bool isinfinityf()
{ return x!=0 && x == 2*x; }
\end{verbatim}

\begin{itemize}
\item floating point
\begin{itemize}
\item Abstract type
\item Why we need small number?
We need computation with small number
\begin{verbatim}
if(x == y && x-y==0) // does this always true????
\end{verbatim}
\end{itemize}
\end{itemize}
\end{itemize}
\end{itemize}

\subsubsection{Type equivalence: Structurally Equivalence and Name equivalence}
\label{sec:orgd86577b}
\begin{itemize}
\item C uses structural equivalence for typedef and name equivalence for struct
type.
\end{itemize}

\subsubsection{Subtype}
\label{sec:org5bcbfd8}
\begin{verbatim}
var a int,
var b 1..127,
\end{verbatim}
\begin{itemize}
\item Subtype in C
\begin{itemize}
\item float and double
\item char* vs. char const* ???
\begin{verbatim}
char *p;
char const* c_p;
c_p = p;
// p = c_p
\end{verbatim}
\end{itemize}
\end{itemize}


\subsection{Polymorphism}
\label{sec:org5d2bc44}
\begin{itemize}
\item Early polymorphism: Type Coersion \& overloading
int vs. unsigned int
\begin{verbatim}
int i = random();
if( i < -217473848)
{
   printf("machine error");
}
\end{verbatim}

long vs. double

multiple bugs
double f(double, int);
double f(int, double);
f(3,5);

Error: Semantically ambiguous

\item Parametric polymorphism
\begin{itemize}
\item iterator
Suppose a list of string, we want to remove the short ones

s = iterator.next
s.length < xxx
s.remove()

This code has some problems because collections might contian any object
so first assigning an object to string may cause trouble.

If we know we are working with list of string, we can do runtime cast.
s = (string) iterator

However, some language has static type checking.
Also, runtime type check has may harm performance. Python has the
"runtime type checking" problem.

In java, we use java template to solve this problem.

c is a collection of string Collection<string>

for(iterator<string> = c.begin(), \ldots{})
   String s = c.next()

No runtime type check.

ML, Java: Generic
C++: templates

Template method is different from java/ml method. In template, we
instantiate the template with the actual type. We know what the type is
at compile time. It is similiar to macros. We actually compitle the code
n times if we have n template. The machine code is copied n times also.

That not how generic work. In Java, we just compile the generic version
once. It is going to think is that going to work with any type. We just
have one copy of the machine code. 

Generic wins?

"sizeof T" with T some template type. The compiler can convert it into a
constant. However, in generic, this is still a variable. We do not know
the size.

Another example
T x,y;
x = y;

In template, we know what instruction to execute because the type is know
at compile time. However, in generic we do not know which instruction to
take. 

However, Java or ML use some other method to make this efficient. Every
object is a pointer. We will never see the actual object. The above
assignment is just poiner assignment. (Java does not have "sizeof"
operator either). That's how generic work. We do not need to know the
exact type of our object. We are just dealling with pointers(references).

\item Subtypes and generic
Soppose a list of string:
List<string> ls = [                  ];
List<Obejct> lo = ls;
lo.add new Thread;
String s = ls.get() // we want the most recently added object

The problem is that although string is a subtype of object, list of
string is not a subtype of list of object.

\begin{verbatim}
void printALL(Collection<?> c) {
  for(object o: c) {
    Object o = c.next();
    System.out.println(o);
  }
}

// Soppose we String has a subtype of BlueString
printAll(Collection<BlueString> cbs); // Does this work?
				      // No because collection of bluesring is not a subtype of collection os string
void printAll<Collection(? extends String> o);
// Only accept types that extends string
\end{verbatim}
\begin{verbatim}
public static void <T> convert( T[], Collection<T> c) {
    for(int i=0; i<a.len ; i++)
    c.add(a[i]);
}

/**
 ** Suppose: BlueString[] a, Collection<String> c will not work on convert
 ** because convert require two parameter to be the same type.
 **/

public static void <T> convert( T[], Collection<? super T> c) {
    for(int i=0; i<a.len ; i++)
    c.add(a[i]);
}
\end{verbatim}

\item Python \& Subtype
\begin{itemize}
\item Duck Typing: 
Never ask whether a type is the subtype of another type. Just check if
it works.
\end{itemize}
\end{itemize}
\end{itemize}

\subsection{Java -- history}
\label{sec:orgd925bc5}
\begin{itemize}
\item What we had
workstations + servers + operation staff on a network = (Solaris written in
C)
\item what they wanted: IoT
\begin{itemize}
\item Reliability
\item size of program
\item Flexibility(OOP)
\item architecture agnostic
\end{itemize}

\item First trial: C++
C++ does provide flexibility but not other thing. For example, template
increases program size.

\item Steal Smalltalk
\begin{itemize}
\item OOP -> Flexibility
\item Bytecodes -> architecture agnostic
Took program and translate it into machine independent bytecode. Then the
interpreter will analyze the bytecode.
\item Runtime checking -> Reliability
\end{itemize}

\item Smalltalk bad ideas
\begin{itemize}
\item Weird syntax
\item Program internally use secret bytecode
For internet of thing, they need download bytecodes instead of machine
code and let the hardware to intepret.
\end{itemize}

\item Oak ==> Java
\begin{itemize}
\item 1995 Browser in Java
HotJava and applets(java code compiled into bytecode)
\item And Browser Mosaic in C++
which becomes IE and Mozilla
Plugin technology which are dynamic linked machine code.
To fix security issue, they later use javascript.
\item Java -> server worked well
-> thread support
\end{itemize}
\end{itemize}


\subsection{Java Types}
\label{sec:org3bda141}
\begin{itemize}
\item Primitive type
byte, short, int, long, char, boolean, float, double
 8      16    32    64   16      1       32     64
				      IEEE-754 fp

In java, the those numbers are fixed for portable flexibility.
In C, performance is more concerned. Machine are allowed to choose more
suitable size.

\begin{itemize}
\item Overflow??
In C, overflow is undefined.
In Java, overflow causes wrapping around ==> reliability.
\end{itemize}

\item References type
Array.
All objects are represented in pointers. You cannot see them but they're
there.

\begin{itemize}
\item Some thing valid in C++ but not in Java
\begin{verbatim}
int foo(int n) {
  char buf[512];

  return ...;
}
\end{verbatim}

Cannot allocate on the stack. Everything is on the heap.
The size is dynamically chosen.
Fixed after created.
\begin{verbatim}
int foo(int n){
    char[] bof = new char[512];
    return xxx;
}

char[] foo(int n) {
    char[] bof = new char[512]; // Could be new char[n];
    return bof;
}
\end{verbatim}
\end{itemize}
\end{itemize}

\subsection{Java: Single Inheritance}
\label{sec:org0012460}
\begin{itemize}
\item Simple and elegant but not flexible
\item Use interfaces instead
\end{itemize}

\section{Week 5}
\label{sec:orgba48538}

\subsection{Abstract Class}
\label{sec:org225327e}

\subsubsection{Java Standard Library}
\label{sec:orga4f52a6}
\begin{enumerate}
\item Collection framwork
Three interfaces
  (Copy)
  Serializable
  RandomAccess

Collection of iterable which could be put into a for loop
  (Abstract collections) which implement collection
  List
    Abstract List
      Abstract Sequential list
	LinkedList  ==> (implement) Queue CS
	Array List CSR
	vector (synchronized, thread-safe compared to array list) CSR
  Set 
    Sorted Set
    Abstract Set 
      Hash set (real class. can use the "new" keyword) CS
	LinkedHashSet
      TreSet CS

\item Object
\begin{itemize}
\item Has constructor 
so we can use "new" keyword for Object.
Can be used as test case
Placeholder

\item public boolean equal(Object arg);  == (reference equality)

\item public int hashCode();  default hashcode is some bits extracted from mem addr.
For building hash table.
We want the following to be true:
If o1 \texttt{= o2 then hashCode(o1) =} hasCode(o2)
Note that the reverse is not true.

\item public final Class<? extends Object> getClass(); 
abstract (class) ==> subclass is mandatory
final (class) ==> subclass is not allowed
final (function) ==> cannot override it.
Final allows inline and optimization ????

Object o = ???
Class c = o.getClass(); ==> May get "Thread"
This is runtime type of o

Reflection:
Program tries to understand itself

What is the point of getClass()?
debug

The "final" keyword does not allow overriding because you could write it in
a way that it return a different class.

\item public String toString();
For debug

\item protected void finalize(); can throw exception
For garbage collector
Why we need to override finalize method?
A subclass overrides the finalize method to dispose of system resources or
to perform other cleanup.

\item protected Object clone();
\end{itemize}

\item class Class
\begin{itemize}
\item public final Class<? extends Object> getClass()
\end{itemize}
\end{enumerate}


\begin{enumerate}
\item Thread
Threads stay out of trouble via:
\begin{itemize}
\item Synchronized
int i;
public synchronized int m() \{ i++; \}
this.lock() -> this.unlock()
\item Object.wait()
\begin{enumerate}
\item remove all synchronized locks held by this thread
\item Wait until the object beomes available
\end{enumerate}

\item Object.notify()
\begin{enumerate}
\item wakes up one thread
\end{enumerate}
\end{itemize}
\end{enumerate}
\end{document}
